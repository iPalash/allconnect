\documentclass[10pt]{article}
\usepackage[margin=0.5in]{geometry}
\usepackage{mathtools}


\begin{document}

\title{CSL333 Homework 2}
\author{\textsc{Kunal Singhal 2012cs10231 } \\  \textsc{ Swapnil Palash 2012cs10257}}
\date{\today}
\maketitle

\section{Broad Idea} MinMax with Alpha-Beta Pruning and Cut-off.

\section{Brief description}
\begin{itemize}
\item Running a MinMax with just Alpha Beta Pruning isn't feasible on 8x9 AllConnect Game, so a depth limited approach is adopted with depth being decide on the time left for the player at each move.
\item The evaluation function uses multiple features, with weights being assigned to them on the basis of intuition and statistical test on playing with other players.
\item The features include:
\begin{itemize}
\item The difference between the k-length sequences of the bot and the opponent's.
\item The difference between the unobstructed (k-1)-length sequences of the bot and the opponent's.
\item The difference between the unobstructed-from-both-side (k-1)-length sequences of the bot and the opponent's.
\end{itemize}
\item Also, the Player essentially has three modes, Attack,defense and Neutral. At the start, it plays aggresively, turns neutral in the middle and at the end plays defensively. 
\end{itemize}

\section{Implementational Details} 
\begin{itemize}
\item The MinMax is implemented in a recursive way with its depth being controlled by a logarithmic function, which is again dependent on the time left for the bot.
\item The Bot Play-mode is implemented by giving weights to the difference in the features of the bot and the opponent's sequences.
\end{itemize}
\section{Running Time}
\begin{itemize}
\item Random Restart = $O(B\cdot R)$
\item One iteration of beam search = $O(B^2 + min(B,R,C)\cdot B \cdot R)$
\end{itemize}
\section{Why This?}
We tried including more features like unobstructed (k-2),(k-2) length sequences but they deteriorated the performance of the player. So,we stuck with the configuration that is presented before you.
\end{document}